\documentclass[french, 12pt, titlepage]{article}
\usepackage{graphicx}
\usepackage{ucs}
\usepackage[utf8]{inputenc}
\usepackage[french]{babel}
\usepackage[T1]{fontenc}
\author{Yoann Thomann\\Mohammed Bouabdellah\\\scriptsize ythomann@univ-mlv.fr\\\scriptsize mbouabde@univ-mlv.fr}
\date{12/04/2010}
\title{\Huge Documentation Indexeur de texte}

\begin{document}
\vspace{\fill}
\maketitle
\newpage
\tableofcontents
\newpage
\section{Description générale}
Ce programme permet, à partir d'un texte, de représenter un dictionnaire en utilisant une table de hachage. Il indexe notamment des mots, en les associant 
à toutes les positions des phrases dans lequels ils peuvent apparaître dans le texte. \\\\
Il est donc possible en executant ce programme, pour un mot, de tester :
\begin{itemize}
\renewcommand{\labelitemi}{$\bullet$}
\item l'appartenance d'un mot au texte
\item l'affichage des positions d'un mot dans le texte
\item l'affichage des phrases contenant un mot
\item l'affichage de la liste triée des mots et de leur position
\item l'affichage des mots commençant par un préfixe donné
\item la sauvegarde de l'index
\end{itemize}

\section{Manuel Utilisateur}


\section{Manuel Developpeur}


\end{document}
