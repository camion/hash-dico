\documentclass[french, 12pt, titlepage]{article}
\usepackage{graphicx}
\usepackage{ucs}
\usepackage[utf8]{inputenc}
\usepackage[french]{babel}
\usepackage[T1]{fontenc}
\usepackage{listings}
\usepackage{color}
\usepackage[colorlinks,linkcolor=black]{hyperref}

\definecolor{light-gray}{gray}{0.95}

\lstset{ %
  language=C,
  basicstyle=\footnotesize,
  numbers=left,
  numberstyle=\footnotesize,
  stepnumber=2,
  numbersep=5pt,
  backgroundcolor=\color{light-gray},
  showspaces=false,
  showstringspaces=false,
  showtabs=false,
  frame=single,
  tabsize=4,
  captionpos=b,
  breaklines=true,
  breakatwhitespace=false,
  escapeinside={\%*}{*}}


\author{Yoann Thomann\\Mohammed Bouabdellah\\\scriptsize ythomann@univ-mlv.fr\\\scriptsize mbouabde@univ-mlv.fr}
\date{12/04/2010}
\title{\Huge Documentation\\Indexeur de texte}

\begin{document}
\vspace{\fill}
\maketitle
\newpage
\tableofcontents
\newpage
\section{Description générale}
Ce programme permet, à partir d'un texte, de représenter un
dictionnaire en utilisant une table de hachage. Il indexe notamment
des mots, en les associant à toutes les positions des phrases dans
lequels ils peuvent apparaître dans le texte. \\\\
Il est donc possible en executant ce programme, pour un mot, de
tester:
\begin{itemize}
\renewcommand{\labelitemi}{$\bullet$}
\item l'appartenance d'un mot au texte
\item l'affichage des positions d'un mot dans le texte
\item l'affichage des phrases contenant un mot
\item l'affichage de la liste triée des mots et de leur position
\item l'affichage des mots commençant par un préfixe donné
\item la sauvegarde de l'index
\end{itemize}

\section{Manuel Utilisateur}
Il existe deux manières d'éxecuter ce programme :
\begin{lstlisting}
./Index [fichier]
\end{lstlisting}
et
\begin{lstlisting}
./Index [option] [mot] fichier
\end{lstlisting}
Ces informations sont accessible a partir de l'executable via la
commande:
\begin{lstlisting}
USAGE
\end{lstlisting}
La première commande donne accès à un menu permettant de manipuler les
diverses fonctions de l'index.\\
La deuxième permet d'effectuer des opérations ponctuelles selon les
arguments de la commande, cette deuxieme est particulièrement adaptée
lors de l'utilisation du programme dans un script.\\
L'utilisateur pour ce faire peut rediriger la sortie standard, puis la
traiter\dots


\begin{lstlisting}
-*-*-*-*-*-*-*-*-*-*-*-*-*-*-*-*-*-*-*-*-*-*-*-*-*-*-*-*-*-*-*-*-*-*-*-*-*
SYNOPSIS:
Index [option] file
        or
Index
-*-*-*-*-*-*-*-*-*-*-*-*-*-*-*-*-*-*-*-*-*-*-*-*-*-*-*-*-*-*-*-*-*-*-*-*-*
Examples:
Index -a word file      | Check if word is in file.
Index -p word file      | Print word positions in file.
Index -P word file      | Print sentences containing word in file.
Index -l text           | Print sorted list of text's words.
Index -d word file      | Print words begining with word in the text.
Index -D out  file      | Save sorted list of file's words in out.DICO
Index -h out  file      | Print this help
-*-*-*-*-*-*-*-*-*-*-*-*-*-*-*-*-*-*-*-*-*-*-*-*-*-*-*-*-*-*-*-*-*-*-*-*-*
\end{lstlisting}

\subsection{Conseils d'utilisation}
La principale recomandation d'utilisation est pour le cas de la
recherche de la présence d'un mot dans le texte en mode interractif,
si l'utilisateur souhaite effectuer d'autres opérations, qui elles
nécessitent un hash, il est recommandé de les faire avant, ce qui
permet de chercher le mot dans le hash. Si l'opération est faite en
premier, le mot sera directement recherché dans le hash.

\section{Manuel Developpeur}


\end{document}
