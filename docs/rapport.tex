\documentclass[french, 12pt, titlepage]{article}
\usepackage{graphicx}
\usepackage{ucs}
\usepackage[utf8]{inputenc}
\usepackage[french]{babel}
\usepackage[T1]{fontenc}
\usepackage{listings}
\usepackage{color}
\usepackage[colorlinks,linkcolor=black]{hyperref}

\definecolor{light-gray}{gray}{0.95}

\lstset{ %
  language=C,
  basicstyle=\footnotesize,
  numbers=left,
  numberstyle=\footnotesize,
  stepnumber=2,
  numbersep=5pt,
  backgroundcolor=\color{light-gray},
  showspaces=false,
  showstringspaces=false,
  showtabs=false,
  frame=single,
  tabsize=4,
  captionpos=b,
  breaklines=true,
  breakatwhitespace=false,
  escapeinside={\%*}{*}}


\author{Yoann Thomann\\Mohammed Bouabdellah\\\scriptsize{ythomann@univ-mlv.fr}\\\scriptsize{mbouabde@univ-mlv.fr}}
\date{12/04/2010}
\title{\Huge Documentation\\Indexeur de texte}

\begin{document}
\vspace{\fill}
\maketitle
\newpage
\tableofcontents
\newpage
\section{Description générale}
Ce programme permet, à partir d'un texte, de représenter un
dictionnaire en utilisant une table de hachage. Il indexe notamment
des mots, en les associant à toutes les positions des phrases dans
lequels ils peuvent apparaître dans le texte. \\\\
Il est donc possible en executant ce programme, pour un mot, de
tester:
\begin{itemize}
\renewcommand{\labelitemi}{$\bullet$}
\item l'appartenance d'un mot au texte
\item l'affichage des positions d'un mot dans le texte
\item l'affichage des phrases contenant un mot
\item l'affichage de la liste triée des mots et de leur position
\item l'affichage des mots commençant par un préfixe donné
\item la sauvegarde de l'index
\end{itemize}

\section{Manuel Utilisateur}
Il existe deux manières d'éxecuter ce programme :
\begin{lstlisting}
./Index [fichier]
\end{lstlisting}
et
\begin{lstlisting}
./Index [option] [mot] fichier
\end{lstlisting}
Ces informations sont accessible a partir de l'executable via la
commande:
\begin{lstlisting}
yoann@hp_laptop# ./Index -h

SYNOPSIS:
Index [option] file
        or
Index [file]

Examples:
Index -a word file      | Check if word is in file.
Index -p word file      | Print word positions in file.
Index -P word file      | Print sentences containing word in file.
Index -l text           | Print sorted list of text's words.
Index -d word file      | Print words begining with word in the text.
Index -D out  file      | Save sorted list of file's words in out.DICO
Index -h out  file      | Print this help
\end{lstlisting}

La première commande donne accès à un menu permettant de manipuler les
diverses fonctions de l'index.\\
La deuxième permet d'effectuer des opérations ponctuelles selon les
arguments de la commande, cette deuxieme est particulièrement adaptée
lors de l'utilisation du programme dans un script.\\
L'utilisateur pour ce faire peut rediriger la sortie standard, puis la
traiter\dots

\subsection{Conseils d'utilisation}
Le fichier passé en argument doit être un fichier texte. Il est possible lors du
traitement du texte de voir l'evolution du traitement en utilisant l'option
verbose \textbf{-v}.\\
Example:
\begin{lstlisting}
./Index -pv oreste Andromaque.txt
\end{lstlisting}

\subsection{Recommandation}
La principale recomandation d'utilisation dans le cas du test d'appartenance
 d'un mot dans le texte en mode interactif, un hash est
 systématiquement éffectué dans ce mode, ainsi si l'utilisateur
 souhaite uniquement rechercher la présence d'un mot dans le texte, il
 serait plus efficace d'utiliser la commande:
\begin{lstlisting}
./Index -a word Colomba.txt
\end{lstlisting}
En effet, cette commande n'engendre elle pas la création d'un hash,
elle recherche directement dans le fichier en le parcourant alors
qu'en mode interractif, un hash du fichier est crée et le mot est
recherché dans celui çi.

\newpage
\section{Manuel Developpeur}
\subsection{Parseur de mot}
Pour récuperer les mots du textes, on aurait pu utiliser une fonction
du type \textit{fgets/fscanf}, mais ces fonctions 'voient' un
mot est comme une suite de caractère séparés par des espaces, retour
chariot, tabulation; ce qui ne correspond pas à notre définition d'un mot.\\
En effet, un mot est pour nous une suite de caractères espacés par des
espaces, retour chariot, tabulation, et divers caractères de
ponctuations. Il nous a donc fallu récuperer les mots caractère
par caractère.\\
Pour la récupération de mots, on peut distinguer 3 cas :
\begin{itemize}
\item fin de mot
\item fin de phrase
\item fin de texte
\end{itemize}

Face à ces 3 cas, on agit de la sorte : si une fin de mot se présente,
on stocke le mot dans la table de hashage ou met a jour sa position
(si déjà présent), si une fin de phrase se présente, on met à jour la
valeur de l'offset (position dans le texte en octet); et enfin dans le
cas d'une fin de texte on s'arrête là.\\\\

\subsection{Parseur de mot}
Le parseur de mot est assez important car il ne récupère que des mots
qui possèdent des caractères alphanumériques. Des suites de caractères
comme par exemple '\textit{***}','\textit{...}' ne sont pas considerés
comme mots.\\
De plus, le parseur à la faculté de placer le curseur du fichier au
bon endroit pour la lecture du prochain mot, en detectant la présence
d'un caractère de fin de phrase après le mot. Ainsi dans la chaîne
\textit{'Bonjour  !'}, le mot \textit{Bonjour}, suivi d'un espace, est
detecté comme étant un mot de fin de phrase, et non pas comme un mot
courant d'une phrase.
\newpage
Ce parseur correspond dans ce programme à la fonction, dont voici le
prototype :

\begin{lstlisting}
int get_word(FILE* stream, char* dest, size_t n);
\end{lstlisting}
Valeurs de retour :
\begin{itemize}
\item EOF   : Fin de fichier
\item 1     : Fin de mot
\item 2     : Fin de phrase
\end{itemize}

Paramètres :
\begin{itemize}
\item stream    : flux dans lequel on récupère nos mots (bien ouvert
  au préalable)
\item dest      : chaîne de caractère dans lequel le mot sera stocké
\item n         : taille de dest, pour eviter des debordements
\end{itemize}

\subsection{Structures utilisées}
Le scheima général de l'architecture structurelle est le suivant.
\\
\includegraphics[width=\textwidth]{schema_projet.jpg}
\\

Dans ce programme, on a besoin de représenter des mots. Pour ce faire,
On définit un mot comme étant une chaîne de caractères accompagnée
d'une liste de positions des phrases dans lequel il apparaît.\\
On utilise donc la structure suivante :
\begin{lstlisting}
typedef struct cellword{
    char *word;
    Lispos l;
}Cellword;
\end{lstlisting}

Une liste de position est définie comme étant une position entière
accompagnée d'un pointeur vers la position suivante.

\begin{lstlisting}
typedef struct cellpos{
	long position;
	struct cellpos* next;
}Cellpos,*Listpos;
\end{lstlisting}

On a donc la posibilité de représenter un mot accompagné de la liste
de ses positions. Il nous faut maintenant, pour manipuler des listes
de mots, une structure permettant de lier ces mots entre eux.\\
Cette structure doit contenir un mot du type que l'on vient de
définir, et un pointeur vers le prochain mot.

\begin{lstlisting}
typedef struct cell{
	Cellword *value;
	struct cell* next;
}Cell,*List;
\end{lstlisting}

Pour detecter la fin d'une phrase ou la fin d'un mot, deux macros ont
été definies :

\begin{lstlisting}
#define END_OF_WORD(c) ((c==' ' || c==',' || c=='\n' || c=='\t' || c==';' || c=='*' || c=='-' || c=='\''))
#define END_OF_PHRASE(c) ((c=='.' || c=='?' || c=='!'))
\end{lstlisting}

Il est donc \textbf{très simple} pour un programmeur d'ajouter ses caractères
de fin de mots/phrases.

\subsection{Hachage ouvert}
Pour stocker nos mots, on utilise un hachage ouvert. On a donc besoin
d'un tableau assez grand qui contiendra des listes de mots. Il est usuel
aussi pour mieu répartir le hachage d'utiliser \textbf{un nombre
  premier} comme taille de ce tableau.\\
Pour insérer un mot dans le tableau on utilise une fonction de hachage
qui, à partir d'une chaîne de caractère, calcule une empreinte
permettant de situer le mot dans notre table. La liste dans laquelle
doit être comtenue mot est donc:\\
\textit{hashTable[hash(mot)]}.\\
L'avantage d'utiliser une table de hachage est que l'accès à un élement
se fait en temps constant, avec un petit temps d'amortissement
(parcours de la liste de mots).\\\\

Les cases de notre table contiennent donc des listes de mots
triés. L'insertion dans notre table correspond à une insertion dans
une liste dont la taille, si le tableau est assez grand, est majorée par une
constante. L'insertion dans notre table de hachage possède donc une
complexité constante.

\subsection{Gestion des listes}
Les listes de mots doivent être triées. L'insertion se fait donc en
respectant l'ordre lexicographique. Etant donné que l'on dispose d'une
liste simplement chainée, l'insertion est donc complexe. En effet, il
faut deux curseurs parcourant la liste en stockant la cellule courante
ainsi que la cellule précedente. Voici le prototype de la fonction
effectuant cette insertion :

\begin{lstlisting}
void insert_lexico_word(List *w, char* word, long offset);
\end{lstlisting}

Et pour éviter une complexité importante, cette fonction a le bon goût
de rechercher le mot à ajouter pour eviter l'insertion de doublon et
de trouver la bonne place où mettre le mot, et ce en un seul passage
dans la liste.\\\\

En ce qui concerne la liste des position, on effectue des insertions
en tête de liste. En effet, vu qu'il n'y a pas de contraintes sur
l'ordre des positions on n'a pas besoin de reparcourir toute la liste.

\subsection{Création de la liste triée}
Après hachage complet du texte, on possède une multitude de listes
triées dans le tableau de hash. Pour obtenir une seule liste triée, on
doit alors fusionner toutes ces listes.\\

On cree alors une liste parallèle, on alloue \textbf{que} de quoi stocker
un mot (Cellword) et un pointeur vers les mot suivant.\\\\

\textbf{ATTENTION :} On ne doit pas creer une nouvelle liste complète,
mais seulement une liste parallèle pointant sur les mots contenus dans
la table. En effet cela reviendrait relativement cher en place et en
temps d'allouer une nouvelle liste contenant des données déjà
existantes. De plus, \textbf{il ne faut pas non plus fusionner directement les
liste existantes entres elles}. En effet, cela effectuera un
remaniements des liens entre les cellules et nous donnera bien une
liste triée, mais aura aussi pour effet de corrompre notre table de
hachage, puisque des mots possédant une clé X pourront être présent
dans une liste de mots possédant une clé Y, ce qui est totalement à
l'encontre du principe de hachage :

\begin{center}
Soit $f$ une fonction de hachage et $E$ ensemble des mots du test:
\\ $\forall(x,y) \in E^{2}, (x \ne y \Rightarrow f(x) \ne f(y))$\\
Une telle fonction doit être injective et donc ne pas retourner une
même clé pour deux mots différents.
\end{center}

On a donc une liste pointant sur le champs \textit{list->value} des
mots existant.\\
Enfin, pour creer notre liste, on parcours notre table et fusionne
toutes les listes avec celle créee en parallèle.\\
Cette fusion correspond dans ce programme aux fonctions, dont voici
les prototypes :

\begin{lstlisting}
void merge_list(List *w1, const List w2);
List create_sorted_list(List hash[]);
\end{lstlisting}

La fonction \textit{create\_sorted\_list} parcours la table de hachage
et fusionne en appellant \textit{merge\_list} toutes les listes de la
table, puis renvoie cette liste triée.

\subsection{Contrôle qualité}
Une fois le programme fonctionnel, des tests ont été effectués, afin
de vérifier et corriger des éventuels \textit{bugs}.\\
Nous avons dans un premier temps utilisé le programme valgrind qui est
un outil de debugging et de profiling.

\begin{verbatim}
valgrind ./Index -vp camion Colomba.txt
...
==4255== HEAP SUMMARY:
==4255==     in use at exit: 0 bytes in 0 blocks
==4255==   total heap usage: 10,043,999 allocs, 10,043,999 frees, 168,883,240 bytes allocated
...
==4255== ERROR SUMMARY: 0 errors from 0 contexts
\end{verbatim}

On constate que a la fin de l'exécussion, la totalité de la mémoire
utilisée est bien libérée, et que l'on a aucune erreur.\\\\

Dans un deuxième temps nous avons testé les \textit{entrées}
en utilisant le fichier \textbf{/dev/random}.

\begin{lstlisting}
yoann@hp_laptop# ./Index -p tt /dev/random
^C
ctrl-c caught, exiting...
\end{lstlisting}

Apres un certain temps, on entre au clavier \textit{ctrl-c} afin
d'arréter le processus. On constate ici que l'on a pas rencontré
d'erreurs ni d'erreur mémoire \textit{segmentation fault}.


\end{document}
